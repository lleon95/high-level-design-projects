%%%%%%%%%%%%%%%%%%%%%%%%%%%%%%%%%%%%%%%%%%%%%%%%%%%%%%%%%%%%%%%%%%%%%%%%%%%%%%%
% Author:  Jorge Castro-Godínez
%
% Chair for Embedded Systems (CES)
% Karlsruhe Institute of Technology
%
% Escuela de Ingeniería Electrónica
% Instituto Tecnológico de Costa Rica
%
% macros.tex
% Macros for papers
% 
% email:	jorge.castro-godinez@kit.edu
%					jocastro@tec.ac.cr
%
%%%%%%%%%%%%%%%%%%%%%%%%%%%%%%%%%%%%%%%%%%%%%%%%%%%%%%%%%%%%%%%%%%%%%%%%%%%%%%%


%------------------------------ PACKAGES --------------------------------------%
%\usepackage{mdwtab}
%\usepackage{tabularx, booktabs}
\usepackage{multirow}

% http://tex.stackexchange.com/questions/26521/
% how-to-change-the-spacing-between-figures-tables-and-text
\setlength{\textfloatsep}{10pt plus 1.0pt minus 2.0pt}

% To add TikZ/PGFPlots
\usepackage{tikz}
\usepackage{pgfplots}
\usepackage{pgf}
\usepackage{caption}
\usepackage{subcaption}
\usepgfplotslibrary{groupplots}
\usetikzlibrary{arrows,automata,shapes,patterns}
\usetikzlibrary{matrix}
%\pgfplotsset{compat=1.8}
%\pgfplotsset{every tick label/.append style={font=\scriptsize}}
\pgfplotsset{every tick label/.append style={font=\scriptsize}}
\usetikzlibrary{shadows}
\usepackage{tikzscale}

\usepackage[spanish, es-noshorthands]{babel}

\usepackage{pgfplotstable}
\pgfplotsset{compat=1.11,
        /pgfplots/ybar legend/.style={
        /pgfplots/legend image code/.code={%
        %\draw[##1,/tikz/.cd,yshift=-0.25em]
                %(0cm,0cm) rectangle (3pt,0.8em);},
        \draw[##1,/tikz/.cd,bar width=3pt,yshift=-0.2em,bar shift=0pt]
                plot coordinates {(0cm,0.8em)};},
			},
}

\usepackage{paralist}
\usepackage{amsmath}
\usepackage{graphicx}

% For wrapping images
\usepackage{wrapfig}

% For algorithms
%\usepackage{algorithm}
%\usepackage{algorithmic}
%
%\floatname{algorithm}{Algorithm}
%\newcommand{\algorithmicinput}{\textbf{Input:}}
%\newcommand{\INPUT}{\item[\algorithmicinput]}
%\newcommand{\algorithmicoutput}{\textbf{Output:}}
%\newcommand{\OUTPUT}{\item[\algorithmicoutput]}
%\newcommand{\algorithmicprocedure}{\hskip0.2in \textbf{Procedure:}}
%\newcommand{\PROCEDURE}{\item[\algorithmicprocedure]}

\usepackage{amsmath}
\usepackage{algorithm}
\usepackage[noend]{algpseudocode}

\newcommand{\sfunction}[1]{\textsf{\textsc{#1}}}
\algrenewcommand\algorithmicforall{\textbf{foreach}}
\algrenewcommand\algorithmicindent{.8em}

%\usepackage{caption}
%\usepackage{subcaption}
%\usepackage{subfigure}

% Colors package from KIT
\usepackage{KITcolors}

% Text in color
\usepackage{color}

% For tables
\usepackage{booktabs}

% For adding comments
\newcommand{\shafique}[1]{\textcolor{red}{Shafique: #1 :Shafique}}
\newcommand{\jorge}[1]{\textcolor{blue}{J: #1}}


% For adding the afiliations
\newcommand{\superscript}[1]{\ensuremath{^{\textrm{#1}}}}
\def\tec{$^1$}
\def\ces{$^2$}
\def\cestec{\superscript{1,2}}

%\def\ces{\superscript{*}}
%\def\caretech{\superscript{\dag}}


\newcommand\encircle[1]{%
  \tikz[baseline=(X.base)] 
    \node (X) [draw, shape=circle, inner sep=0] {\strut #1};}


% Subfigures
%\usepackage[caption=false,position=bottom]{subfig}
%\usepackage[justification=Centering]{subfig}
%\usepackage[]{subfig}
%\usepackage{subfigure}
%\usepackage{subcaption}


% For code listings
%\usepackage{listings}

\usepackage[]{minted}
%\usepackage[chapter, newfloat=true]{minted}
\usemintedstyle{default}
\setminted{fontsize=\scriptsize}


%\lstset{frame=tb,
%  language=C,
%  aboveskip=3mm,
%  belowskip=3mm,
%  showstringspaces=false,
%  columns=flexible,
%  basicstyle={\footnotesize\ttfamily},
%  numbers=none,
%  numberstyle=\tiny\color{gray},
%  keywordstyle=\color{blue},
%  commentstyle=\color{dkgreen},
%  stringstyle=\color{mauve},
%  breaklines=true,
%  breakatwhitespace=true,
%  tabsize=3
%}

% Hyperreferences
%\definecolor{dark-blue}{rgb}{0.15,0.15,0.4}
%\usepackage[colorlinks = true,
%	allcolors = dark-blue,
%	bookmarks = false]{hyperref}

\usepackage{url}

% Enumerate
\usepackage{enumitem}

%\renewcommand{\arraystretch}{1.5}



% Modify text height and line spacing
%\addtolength{\textheight}{22pt}
%\linespread{0.925}

% ---------------------------- COLOR DEFINITIONS ----------------------------- %


% ------------------------------- TIKZ BLOCKS -------------------------------- %


% We need layers to draw the block diagram
\pgfdeclarelayer{background}
\pgfdeclarelayer{foreground}
\pgfsetlayers{background,main,foreground}


% Define a few styles and constants, for diagrams
%\tikzstyle{ann} = [above, text width=5em, text centered]
%\tikzstyle{textnode} = [above, text width=30em]
%\def\blockdist{2.3}
%\def\edgedist{2.5}
%\tikzstyle{mpu}=[draw, fill=RdBu-11-3, text width=5.0em, 
%    text centered, minimum height=5.0em, rounded corners]
%\tikzstyle{accel}=[draw, fill=RdBu-11-9, text width=4.0em, 
%    text centered, minimum height=5.0em, rounded corners]
%\tikzstyle{sdram}=[draw, fill=RdBu-11-4, text width=6.0em, 
%    text centered, minimum height=2.0em]
%\tikzstyle{mem}=[draw, text width=4.0em, 
%    text centered, minimum height=1.5em,dashed]
%\tikzstyle{L3}=[draw, fill=RdBu-11-4, text width=7.0em, 
%    text centered, minimum height=2.0em]
%\tikzstyle{port}=[draw, fill=RdBu-11-4, text width=2.0em, 
%    text centered, minimum height=2.0em]
%\tikzstyle{block}=[draw, fill=black!20, text width=3.0em, 
%    text centered, minimum height=3.0em, rounded corners]



% ------------------------------- CASES -------------------------------------- %
\tikzstyle{adder} = [draw, thick, circle, fill=KITblack15]
\tikzstyle{ops} = [draw, thick, circle, fill=KITblack15]

\tikzstyle{lslr} = [draw, thick, circle, fill=white]
\tikzstyle{add} = [draw, thick, circle, fill=KITgreen30]

\tikzstyle{nodeIO} = [draw=none, fill=none, text centered]
\tikzstyle{tag} = [draw=none, fill=none, text centered]

\tikzstyle{ann} = [above, text width=5em, text centered]
\tikzstyle{textnode} = [above, text width=30em]

\tikzstyle{tool}=[draw, fill=KITblue70, minimum width=7.0em, text centered, 
	text=white]

\tikzstyle{tool2}=[draw, fill=KITblue70, minimum width=6.0em, text centered, 
	text=white]

\tikzstyle{tool3}=[draw, fill=KITblue70, minimum width=5.0em, text centered, 
	text=white, align=center]

\tikzstyle{res}=[draw, minimum width=5.5em, text centered, 
	text=black]

\tikzstyle{database}=[cylinder, cylinder uses custom fill, cylinder body 
		fill=KITgreen30, cylinder end fill=KITgreen30, shape border rotate=90,
		aspect=0.25, draw, minimum height=1.8em, text width=1.3em, text centered]

\tikzstyle{compiled} = [draw, minimum height=6.0mm, minimum width=5.0mm, 
    text centered, fill=KITbrown30] 



%\tikzstyle{tool}=[draw, fill=KITgreen50, text width=4.2em, 
%    text centered, minimum height=3.0em, rounded corners]  
   
%\tikzstyle{tool2}=[draw, fill=KITblue50, text width=4.2em, 
%    text centered, minimum height=3.0em, rounded corners]
    
%\tikzstyle{appSim}=[draw, fill=RdBu-11-9, text width=4.0em, 
%    text centered, minimum height=3.0em, rounded corners]	

% taken from manual
\makeatletter
\pgfdeclareshape{document}{
  \inheritsavedanchors[from=rectangle] % this is nearly a rectangle
  \inheritanchorborder[from=rectangle]
  \inheritanchor[from=rectangle]{center}
  \inheritanchor[from=rectangle]{north}
  \inheritanchor[from=rectangle]{south}
  \inheritanchor[from=rectangle]{west}
  \inheritanchor[from=rectangle]{east}
  % ... and possibly more
  \backgroundpath{% this is new
    % store lower right in xa/ya and upper right in xb/yb
    \southwest \pgf@xa=\pgf@x \pgf@ya=\pgf@y
    \northeast \pgf@xb=\pgf@x \pgf@yb=\pgf@y
    % compute corner of ‘‘flipped page’’
    \pgf@xc=\pgf@xb \advance\pgf@xc by-5pt % this should be a parameter
    \pgf@yc=\pgf@yb \advance\pgf@yc by-5pt
    % construct main path
    \pgfpathmoveto{\pgfpoint{\pgf@xa}{\pgf@ya}}
    \pgfpathlineto{\pgfpoint{\pgf@xa}{\pgf@yb}}
    \pgfpathlineto{\pgfpoint{\pgf@xc}{\pgf@yb}}
    \pgfpathlineto{\pgfpoint{\pgf@xb}{\pgf@yc}}
    \pgfpathlineto{\pgfpoint{\pgf@xb}{\pgf@ya}}
    \pgfpathclose
    % add little corner
    \pgfpathmoveto{\pgfpoint{\pgf@xc}{\pgf@yb}}
    \pgfpathlineto{\pgfpoint{\pgf@xc}{\pgf@yc}}
    \pgfpathlineto{\pgfpoint{\pgf@xb}{\pgf@yc}}
    \pgfpathlineto{\pgfpoint{\pgf@xc}{\pgf@yc}}
  }
}
\makeatother

\makeatletter
\pgfdeclareshape{documentsm}{
  \inheritsavedanchors[from=rectangle] % this is nearly a rectangle
  \inheritanchorborder[from=rectangle]
  \inheritanchor[from=rectangle]{center}
  \inheritanchor[from=rectangle]{north}
  \inheritanchor[from=rectangle]{south}
  \inheritanchor[from=rectangle]{west}
  \inheritanchor[from=rectangle]{east}
  % ... and possibly more
  \backgroundpath{% this is new
    % store lower right in xa/ya and upper right in xb/yb
    \southwest \pgf@xa=\pgf@x \pgf@ya=\pgf@y
    \northeast \pgf@xb=\pgf@x \pgf@yb=\pgf@y
    % compute corner of ‘‘flipped page’’
    \pgf@xc=\pgf@xb \advance\pgf@xc by-4pt % this should be a parameter
    \pgf@yc=\pgf@yb \advance\pgf@yc by-4pt
    % construct main path
    \pgfpathmoveto{\pgfpoint{\pgf@xa}{\pgf@ya}}
    \pgfpathlineto{\pgfpoint{\pgf@xa}{\pgf@yb}}
    \pgfpathlineto{\pgfpoint{\pgf@xc}{\pgf@yb}}
    \pgfpathlineto{\pgfpoint{\pgf@xb}{\pgf@yc}}
    \pgfpathlineto{\pgfpoint{\pgf@xb}{\pgf@ya}}
    \pgfpathclose
    % add little corner
    \pgfpathmoveto{\pgfpoint{\pgf@xc}{\pgf@yb}}
    \pgfpathlineto{\pgfpoint{\pgf@xc}{\pgf@yc}}
    \pgfpathlineto{\pgfpoint{\pgf@xb}{\pgf@yc}}
    \pgfpathlineto{\pgfpoint{\pgf@xc}{\pgf@yc}}
  }
}
\makeatother

\tikzstyle{doc}=[draw, align=center, color=black, shape=document, 
  minimum width=5.0mm, minimum height=6.5mm, shape=document]

\tikzstyle{docsm}=[draw, align=center, color=KITbrown, fill=KITbrown,
  shape=document, minimum width=5.0mm, minimum height=6.0mm, shape=documentsm]



% BLOCKS for CNN diagram

%\tikzstyle{cnn1}=[draw, fill=mywhite, text width=2.8em, 
%    text centered, minimum height=3.4em]
%\tikzstyle{cnn2}=[draw, fill=KITblack30, text width=2.8em, 
%    text centered, minimum height=3.4em]

%\tikzstyle{block2}=[draw, fill=KITblack30, text width=3.0em, 
%    text centered, minimum height=3.0em, rounded corners]
    
    
% BLOCKS HW accelerator

%\tikzstyle{suma} = {draw, fill=KITblue70, circle, node distance=1cm}

%\tikzset{%
%  block/.style    = {draw, thick, rectangle, minimum height = 3em,
%    minimum width = 3em},
%  sum/.style      = {draw, circle, fill=KITblack15, node distance = 1cm}, % 
  %Adder
%  mult/.style     = {draw, circle, node distance = 1cm}, % Multiplier
%  div/.style      = {draw, circle, node distance = 1cm}, % Divider
%  input/.style    = {coordinate}, % Input
%  output/.style   = {coordinate} % Output
%}
%draw, circle, node distance = 2cm

%BLOCKS

% styles for accelerator data path
%\tikzstyle{regs}=[draw, fill=RdBu-11-9, text width=2.0em, 
%    text centered, minimum height=5.5em]
    
%\tikzstyle{loop}=[draw, fill=RdBu-11-9, text width=2.0em, 
%    text centered, minimum height=1.5em]
    
%\tikzstyle{cac}=[draw, fill=RdBu-11-9, text width=4.8em, 
%    text centered, minimum height=2.4em, rounded corners, thick, dashed]


